\documentclass[a5paper]{article}
\usepackage[a5paper, top=17mm, bottom=17mm, left=17mm, right=17mm]{geometry}
\usepackage[utf8]{inputenc}
\usepackage[T2A,T1]{fontenc}
\usepackage[colorlinks,filecolor=blue,citecolor=green,unicode,pdftex]{hyperref}
\usepackage{cmap}
\usepackage[english,russian]{babel}
\usepackage{amsmath}
\usepackage{amssymb,amsfonts,textcomp}
\usepackage{color}
\usepackage{array}
\usepackage{hhline}
\hypersetup{colorlinks=true, linkcolor=blue, citecolor=blue, filecolor=blue, urlcolor=blue, pdftitle=1, pdfauthor=, pdfsubject=, pdfkeywords=}
% \usepackage[pdftex]{graphicx}
\usepackage{graphicx}
% \usepackage{epigraph}
% Раскомментировать тем, у кого этот пакет есть. Шрифт станет заметно красивее.
%\usepackage{literat}
\usepackage{indentfirst}
\usepackage{multirow}
\usepackage{subfig}

\sloppy
\pagestyle{plain}
%\pagestyle{empty}

\title{Метамоделирование: современный подход к созданию средств визуального проектирования}

%\author{ \and Т.А. Брыксин \and Ю.В. Литвинов}
\date{}
\begin{document}

\maketitle
\thispagestyle{empty}

\section*{Введение}
Существуют различные подходы для разработки ПО с помощью средств визуального моделирования. В настоящее время имеются различные подходы к такого рода разработке. Некоторые инструментарии представляют собой набор базовых визуальных языков, необходимых для создания новых систем. Такие инструментарии являются CASE-пакетами и носят общий характер, поскольку не являются ориентированными на конкретную предметную область. Их использование в отдельных случаях упрощает проектирование новых систем. Если же проектируемая система представляет узко специализированную предметную область или является довольно громоздкой, её создание может стать неоправданно довольно трудоёмким процессом. В этом случае уместен другой подход к разработке новых систем - metaCASE-подход. Суть его состоит в том, что предлагается инструментарий для создания специализированных для рассматриваемой предметной области языков, и система проектируется уже с помощью разработанного языка. Такой подход является более гибким и удобным в данном контексте.

Примерами средств для создания новых предметно-ориентированных языков(DSL) являются Microsoft DSL Tools, Eclipse GMF, MetaEdit+.

Инструментарий Microsoft DSL Tools используется для создания редакторов визуальных языков, встроенных в среду Visual Studio, и хорошо применим для создания несложных графических редакторов. Создание нового графического редактора состоит в описании его метамодели с возможностью дополнения требуемой функциональности на C#. Но с помощью него нельзя создать независимый визуальный язык. 

Технология Eclipse GMF разрабатывается на базе среды разработки Eclipse и также предназначена для создания новых предметно-ориентированных языков, в основном встроенных в среду. Но создание метамодели языка в свою очередь требует описания нескольких моделей (доменной, графической модели, моделей инструментов, соответствия и генератора), поэтому процесс создания нового языка является довольно длительным и трудоёмким.

MetaEdit+ является наиболее развитым инструментарием для создания DSM-решений, имеет большое число промышленных внедрений. Но для использования данного инструментария требуется специальная подготовка и цена лицензии на него весьма высока.

Нашей целью была разработка инструментария, сочетающего в себе преимущества существующих и, по возможности, лишенного их недостатков.

\section*{QReal}
...(структура и всё	такое)
Но такой способ создания новых графических редакторов в системе имеет ряд принципиальных недостатков. Во-первых, написание xml-файла - это довольно кропотливый процесс, поскольку описание каждого элемента громоздко и не наглядно. Во-вторых, пользователю необходимо обладать определенными знаниями касаемо строения метаметамодели языка и навыками работы с XML.

\section*{Метаредактор}
В связи с этим было предпринято расширение инструментов поддержки метамоделирования в QReal. Для этих целей был разработан специальный язык для создания новых визуальных языков (так называемый метаязык), базовыми элементами которого являются сущности и отношения. С помощью этого языка можно задавать элементы графического редактора, возможные связи между ними, свойства элементов, отношения наследования между элементами на диаграмме, отношения допустимой вложенности одних элементов в другие и различные дополнительные свойства, поддержка которых осуществлена в QReal (способность “вытягивать” из элементов определенные связи, сортировать вложенные элементы и уметь их скрывать для элементов контейнеров и другое). К тому же, имеется поддержка перечисляемых типов данных, которые можно использовать в данной диаграмме. У одного графического редактора можно создавать несколько диаграмм. 
Для разработанного языка была осуществлена инструментальная поддержка. После построения метамодели (или нескольких метамоделей) языка пользователь имеет возможность конвертировать его в используемый xml-формат. Была создана инфраструктура, обеспечивающая поддержку сквозного процесса создания графических редакторов. С ее помощью разработчик может спроектировать новый визуальный язык, скомпилировать подключаемый модуль соответствующего графического редактора и подключить его к QReal, не выходя из системы. Также кроме этого реализован разбор существующих xml-файлов и их визуализация с помощью метаредактора, тем самым кроме создания новых визуальных языков пользователь может редактировать уже имеющиеся. Тем самым парсер xml-файлов совместно с генератором в xml-формат позволяет использовать и старый способ задания метамодели языка, то есть данные подходы взаимозаменяемы в зависимости от предпочтений пользователя.

\end{document}